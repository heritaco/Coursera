\documentclass[10pt,a4paper]{article}
% \usepackage[spanish]{babel}
% \usepackage[english]{babel}

% --- Layout & links ---
\usepackage[margin=4cm]{geometry}
\usepackage[hidelinks]{hyperref}

% --- Matemáticas ---
\usepackage{amsmath,amssymb}      % Standard AMS math packages

% --- Listas ---
\usepackage{enumitem}
\setlist{nosep}                    % Remove extra vertical space in lists
\setlist[itemize]{label=--}        % Itemize uses en dash as bullet

% Global paragraph style: no indentation, add vertical space between paragraphs
\setlength{\parindent}{0pt}
\setlength{\parskip}{1em}

% --- Fuentes (texto y matemáticas) ---
\usepackage{fontspec}              % Font selection for Unicode engines

% Main text font: EB Garamond
\setmainfont{EB Garamond}[
  UprightFont   = * Regular,
  ItalicFont    = * Italic,
  BoldFont      = * SemiBold,
  BoldItalicFont= * SemiBold Italic
]

% Math font: newtxmath (Times-like math)
\usepackage{newtxmath}

% Replace \mathbb with Libertinus Math blackboard bold
%   bb=libus  -> use Libertinus blackboard-bold
%   bbscaled  -> overall scaling factor
\usepackage[bb=libus,bbscaled=1.0]{mathalpha}


% --- Secciones y subsecciones ---
\usepackage{titlesec}

\newfontface\extrabold{EB Garamond ExtraBold}

\titleformat{\section}
  {\bfseries\Large}{\thesection}{0.5em}{}
\titleformat{\subsection}
  {\bfseries\large}{\thesubsection}{0.5em}{}


% --- Encabezado simple ---
\usepackage{fancyhdr}
\pagestyle{empty}
\fancyhf{}


% --- Espaciado ---
\usepackage{setspace}
\setstretch{1.2}




\begin{document}
% \thispagestyle{empty}
% \vspace*{-5em}
% \begin{center}
%   \LARGE\extrabold{Artificial Intelligence in Algorithmic Trading: An Extended Guide}
% \end{center}
\section*{HPD set for the posterior $\pi(\theta \mid y) = 2\theta$ on $(0,1)$}

\subsection*{Posterior model}

Consider a one-dimensional parameter $\Theta$ with posterior density
\begin{equation}
  \pi(\theta \mid y) =
  \begin{cases}
    2\theta, & 0 < \theta < 1,\\[4pt]
    0,       & \text{otherwise}.
  \end{cases}
\end{equation}
We first check that this is a proper density:
\begin{equation}
  \int_{0}^{1} 2\theta \, d\theta
  = \left[ \theta^{2} \right]_{0}^{1}
  = 1.
\end{equation}

We want the $(1-\alpha)$ highest posterior density (HPD) set $H_{1-\alpha}$ using the \emph{level set} definition:
\begin{equation}
  H_{1-\alpha}
  = \left\{ \theta : \pi(\theta \mid y) \ge k_\alpha \right\},
\end{equation}
where the threshold $k_\alpha$ is chosen such that
\begin{equation}
  \int_{H_{1-\alpha}} \pi(\theta \mid y)\, d\theta
  = 1 - \alpha.
\end{equation}

\subsection*{Level sets of the posterior}

For a fixed level $k > 0$, the level set of $\pi(\theta \mid y)$ is
\begin{equation}
  L(k)
  := \left\{ \theta \in (0,1) : \pi(\theta \mid y) \ge k \right\}
   = \left\{ \theta \in (0,1) : 2\theta \ge k \right\}.
\end{equation}
Solving the inequality $2\theta \ge k$ gives
\begin{equation}
  \theta \ge \frac{k}{2}.
\end{equation}
Intersecting with the support $0 < \theta < 1$ yields, for $0 < k \le 2$,
\begin{equation}
  L(k) = \left[ \frac{k}{2},\, 1 \right].
\end{equation}
(For $k > 2$ the set is empty; for $k \le 0$ the set is $(0,1)$, but the interesting case is $0<k\le2$.)

Thus, for $0 < k \le 2$,
\begin{equation}
  L(k) = \left[ \frac{k}{2}, 1 \right].
\end{equation}

\subsection*{Posterior probability of a level set}

Define the posterior probability contained in the level set $L(k)$ by
\begin{equation}
  \varphi(k)
  := P\bigl( \Theta \in L(k) \mid y \bigr)
  = \int_{L(k)} \pi(\theta \mid y)\, d\theta.
\end{equation}
For $0 < k \le 2$, using $L(k) = [k/2, 1]$, we obtain
\begin{align}
  \varphi(k)
  &= \int_{k/2}^{1} 2\theta \, d\theta \\
  &= \left[ \theta^{2} \right]_{\theta = k/2}^{\theta = 1}
   = 1 - \left( \frac{k}{2} \right)^{2}.
\end{align}
Thus
\begin{equation}
  \varphi(k) = 1 - \left( \frac{k}{2} \right)^{2},
  \qquad 0 < k \le 2.
\end{equation}

\subsection*{Determining the HPD threshold $k_\alpha$}

By definition, the $(1-\alpha)$ HPD set corresponds to the level $k_\alpha$ such that
\begin{equation}
  \varphi(k_\alpha) = 1 - \alpha.
\end{equation}
Using the explicit expression for $\varphi(k)$,
\begin{equation}
  1 - \left( \frac{k_\alpha}{2} \right)^{2}
  = 1 - \alpha,
\end{equation}
which implies
\begin{equation}
  \left( \frac{k_\alpha}{2} \right)^{2} = \alpha
  \quad \Longrightarrow \quad
  k_\alpha = 2\sqrt{\alpha},
\end{equation}
where we take the positive root since $k_\alpha \ge 0$.

\subsection*{HPD region}

The HPD region $H_{1-\alpha}$ is the level set at height $k_\alpha$:
\begin{equation}
  H_{1-\alpha}
  = L(k_\alpha)
  = \left[ \frac{k_\alpha}{2},\, 1 \right]
  = \left[ \frac{2\sqrt{\alpha}}{2},\, 1 \right]
  = [\sqrt{\alpha},\, 1 ].
\end{equation}

Therefore, the $(1-\alpha)$ HPD credible set for this posterior is
\begin{equation}
  H_{1-\alpha}
  = [\sqrt{\alpha},\, 1 ].
\end{equation}

\subsection*{Verification of the posterior mass}

We can check directly that this set has posterior probability $1-\alpha$:
\begin{align}
  P(\Theta \in H_{1-\alpha} \mid y)
  &= \int_{\sqrt{\alpha}}^{1} 2\theta \, d\theta \\
  &= \left[ \theta^{2} \right]_{\theta = \sqrt{\alpha}}^{\theta = 1} \\
  &= 1 - (\sqrt{\alpha})^{2}
   = 1 - \alpha.
\end{align}

\subsection*{Comparison with the equal-tailed credible interval}

The CDF associated with $\pi(\theta \mid y) = 2\theta$ on $(0,1)$ is
\begin{equation}
  F(\theta)
  = P(\Theta \le \theta \mid y)
  = \int_{0}^{\theta} 2t \, dt
  = \theta^{2}.
\end{equation}
A $(1-\alpha)$ equal-tailed credible interval $(a,b)$ satisfies
\begin{equation}
  F(a) = \frac{\alpha}{2}, \qquad
  F(b) = 1 - \frac{\alpha}{2}.
\end{equation}
Thus
\begin{equation}
  a^{2} = \frac{\alpha}{2}
  \quad\Longrightarrow\quad
  a = \sqrt{\frac{\alpha}{2}},
\end{equation}
and
\begin{equation}
  b^{2} = 1 - \frac{\alpha}{2}
  \quad\Longrightarrow\quad
  b = \sqrt{1 - \frac{\alpha}{2}}.
\end{equation}
Hence the equal-tailed $(1-\alpha)$ credible interval is
\begin{equation}
  (a,b)
  = \left( \sqrt{\frac{\alpha}{2}},\ \sqrt{1 - \frac{\alpha}{2}} \right),
\end{equation}
which is clearly different from the HPD region
\begin{equation}
  H_{1-\alpha} = [\sqrt{\alpha},\, 1 ].
\end{equation}

\bigskip

\noindent
\[
  \boxed{
    H_{1-\alpha} \;=\; [\sqrt{\alpha},\, 1] \quad\text{for}\quad
    \pi(\theta \mid y) = 2\theta,\; 0 < \theta < 1.
  }
\]
\section*{Highest Posterior Density (HPD) Sets: General Definition}

\subsection*{Setup}

Let $(\Theta,\mathcal{B})$ be a parameter space (for instance $\Theta \subseteq \mathbb{R}^d$)
with posterior density
\begin{equation}
  \pi(\theta \mid y), \qquad \theta \in \Theta,
\end{equation}
with respect to some reference measure $\mu$ (typically Lebesgue measure), so that
\begin{equation}
  \int_{\Theta} \pi(\theta \mid y)\, d\mu(\theta) = 1.
\end{equation}

We fix a credibility level $1-\alpha$ with $0<\alpha<1$ (e.g. $\alpha = 0.05$ for a
$95\%$ credible level). A \emph{credible set} $C \subseteq \Theta$ of level $1-\alpha$
satisfies
\begin{equation}
  P(\Theta \in C \mid y)
  = \int_{C} \pi(\theta \mid y)\, d\mu(\theta)
  = 1 - \alpha.
\end{equation}

Among all such sets, the \emph{highest posterior density (HPD) set} is the one that
contains the $(1-\alpha)$ most probable parameter values, in the sense of posterior
density.

\subsection*{Level sets of the posterior density}

For each \emph{density level} $k \ge 0$ we define the \emph{level set} of
$\pi(\theta \mid y)$ at height $k$ by
\begin{equation}
  L(k)
  := \left\{ \theta \in \Theta : \pi(\theta \mid y) \ge k \right\}.
\end{equation}
Intuitively, $L(k)$ is the subset of parameter values for which the posterior density
is at least $k$. As $k$ decreases, the set $L(k)$ becomes larger.

We then define the posterior \emph{probability content} of the level set $L(k)$ as
\begin{equation}
  \varphi(k)
  := P(\Theta \in L(k) \mid y)
  = \int_{L(k)} \pi(\theta \mid y)\, d\mu(\theta).
\end{equation}
Since lowering $k$ can only enlarge $L(k)$, the function $\varphi(k)$ is nonincreasing
in $k$:
\begin{equation}
  k_1 > k_2
  \quad \Longrightarrow \quad
  L(k_1) \subseteq L(k_2)
  \quad \Longrightarrow \quad
  \varphi(k_1) \le \varphi(k_2).
\end{equation}
Moreover, as $k \to 0$ we recover the whole parameter space, so
$\varphi(k) \to 1$, while as $k$ increases towards the maximum of $\pi(\theta\mid y)$
we have $\varphi(k) \to 0$.

\subsection*{Definition of the $(1-\alpha)$ HPD set}

The $(1-\alpha)$ HPD set $H_{1-\alpha}$ is defined via a \emph{density threshold}
$k_\alpha$ such that the level set at height $k_\alpha$ contains exactly $(1-\alpha)$
posterior probability:
\begin{equation}
  \varphi(k_\alpha)
  = \int_{L(k_\alpha)} \pi(\theta \mid y)\, d\mu(\theta)
  = 1 - \alpha.
\end{equation}
Whenever $\varphi$ is continuous and strictly decreasing in $k$ over the relevant
range, the value $k_\alpha$ is uniquely determined by this equation.

The $(1-\alpha)$ HPD set is then
\begin{equation}
  H_{1-\alpha}
  = L(k_\alpha)
  = \left\{ \theta \in \Theta : \pi(\theta \mid y) \ge k_\alpha \right\}.
\end{equation}

By construction:
\begin{enumerate}
  \item $H_{1-\alpha}$ has posterior probability
  \begin{equation}
    P(\Theta \in H_{1-\alpha} \mid y)
    = \int_{H_{1-\alpha}} \pi(\theta \mid y)\, d\mu(\theta)
    = 1 - \alpha.
  \end{equation}
  \item For any $\theta_1 \in H_{1-\alpha}$ and $\theta_2 \notin H_{1-\alpha}$,
  we have
  \begin{equation}
    \pi(\theta_1 \mid y) \ge k_\alpha > \pi(\theta_2 \mid y),
  \end{equation}
  i.e. every point inside $H_{1-\alpha}$ has posterior density at least as large
  as any point outside.
\end{enumerate}
This expresses formally the idea that $H_{1-\alpha}$ is the set of ``most probable''
parameter values with total probability $1-\alpha$.

\subsection*{Univariate case and HPD intervals}

In the one-dimensional case ($\Theta \subseteq \mathbb{R}$), $H_{1-\alpha}$ is called
an \emph{HPD interval} when it is connected. If the posterior density $\pi(\theta \mid y)$
is strictly unimodal and sufficiently regular, then for each $k$ the level set $L(k)$ is
typically an interval
\begin{equation}
  L(k) = [a(k), b(k)],
\end{equation}
and the HPD set $H_{1-\alpha}$ is simply the shortest interval centered around the
mode that has posterior probability $1-\alpha$.

If the posterior density is \emph{multimodal}, the level set $L(k)$, and hence the
HPD set $H_{1-\alpha}$, can be a union of disjoint intervals:
\begin{equation}
  H_{1-\alpha}
  = \bigcup_{j=1}^{J} I_j, \qquad
  I_j \subset \Theta,
\end{equation}
each $I_j$ containing one or more modes, with the property that the union has total
probability $1-\alpha$ and consists only of points whose density is at least $k_\alpha$.

\subsection*{Relation to equal-tailed credible intervals}

Let $F(\theta \mid y)$ denote the posterior cumulative distribution function (CDF),
\begin{equation}
  F(\theta \mid y)
  := P(\Theta \le \theta \mid y)
  = \int_{(-\infty,\theta]} \pi(t \mid y)\, dt
  \qquad \text{(univariate case)}.
\end{equation}
A $(1-\alpha)$ \emph{equal-tailed} credible interval $(a,b)$ is defined by the
quantile conditions
\begin{equation}
  F(a \mid y) = \frac{\alpha}{2},
  \qquad
  F(b \mid y) = 1 - \frac{\alpha}{2}.
\end{equation}
Both $(a,b)$ and $H_{1-\alpha}$ satisfy
\begin{equation}
  P(a < \Theta < b \mid y) = 1 - \alpha,
  \qquad
  P(\Theta \in H_{1-\alpha} \mid y) = 1 - \alpha,
\end{equation}
but they need not coincide. They do coincide in many common symmetric and unimodal
cases (e.g.\ Normal posteriors), but in general:
\begin{itemize}
  \item Equal-tailed intervals are determined by \emph{tail probabilities};
  \item HPD sets are determined by \emph{contours of constant density}.
\end{itemize}

\subsection*{Summary}

Given a posterior density $\pi(\theta \mid y)$ and level $1-\alpha$, the HPD set is
defined as
\begin{equation}
  H_{1-\alpha}
  =
  \left\{
    \theta \in \Theta :
    \pi(\theta \mid y) \ge k_\alpha
  \right\},
\end{equation}
where $k_\alpha$ is chosen such that
\begin{equation}
  \int_{\{\theta : \pi(\theta \mid y) \ge k_\alpha\}} \pi(\theta \mid y)\, d\mu(\theta)
  = 1 - \alpha.
\end{equation}
This set has posterior probability $1-\alpha$ and contains only points of highest
posterior density relative to the rest of the parameter space.

\end{document}