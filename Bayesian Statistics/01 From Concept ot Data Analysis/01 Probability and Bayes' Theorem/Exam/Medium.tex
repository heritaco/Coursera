% ======================================================================
%   Document class
%   Compile with: lualatex or xelatex
% ======================================================================
\documentclass[10pt,a4paper]{article}

% Main document language (uncomment one of these):
% \usepackage[spanish]{babel}
% \usepackage[english]{babel}

% --- Define reusable metadata macros ---
\newcommand{\DocTitle}{Artificial Intelligence in Algorithmic Trading: An Extended Guide}
\newcommand{\DocAuthor}{Heriberto Espino Montelongo}
\newcommand{\DocKeywords}{artificial intelligence, algorithmic trading, quantitative finance, machine learning}
% Connect to standard LaTeX title/author
\title{\DocTitle}
\author{\DocAuthor}


% ======================================================================
%   Page layout & hyperlinks
% ======================================================================
\usepackage[margin=4cm]{geometry} % Page margins
\usepackage[hidelinks]{hyperref}  % Clickable links without colored boxes
% --- Hyperref metadata using the same macros ---
\hypersetup{
  pdfauthor={\DocAuthor},
  pdftitle={\DocTitle},
  pdfkeywords={\DocKeywords}
}

% ======================================================================
%   Mathematics and symbols
% ======================================================================
\usepackage{amsmath,amssymb}      % Standard AMS math packages

% ======================================================================
%   Lists (itemize, enumerate) and paragraph layout
% ======================================================================
\usepackage{enumitem}
\setlist{nosep}                    % Remove extra vertical space in lists
\setlist[itemize]{label=--}        % Itemize uses en dash as bullet

% Global paragraph style: no indentation, add vertical space between paragraphs
\setlength{\parindent}{0pt}
\setlength{\parskip}{1em}

% ======================================================================
%   Fonts: text + math
%   Requires LuaLaTeX or XeLaTeX
% ======================================================================
\usepackage{fontspec}              % Font selection for Unicode engines

% Main text font: EB Garamond
\setmainfont{EB Garamond}[
  UprightFont   = * Regular,
  ItalicFont    = * Italic,
  BoldFont      = * SemiBold,
  BoldItalicFont= * SemiBold Italic
]

% Math font: newtxmath (Times-like math)
\usepackage{newtxmath}

% Replace \mathbb with Libertinus Math blackboard bold
%   bb=libus  -> use Libertinus blackboard-bold
%   bbscaled  -> overall scaling factor
\usepackage[bb=libus,bbscaled=1.0]{mathalpha}

% ======================================================================
%   Section and subsection formatting
% ======================================================================
\usepackage{titlesec}

% Optional explicit bold faces for headings (not strictly necessary)
\newfontface\garamondbold{EB Garamond SemiBold}
\newfontface\garamondbolditalic{EB Garamond SemiBold Italic}

% Section title format
\titleformat{\section}
  {\bfseries\Large\clearpage}     % Format for the title text
  {\thesection}         % Section number
  {1em}                 % Space between number and title
  {}                    % Code before the title text

% Subsection title format
\titleformat{\subsection}
  {\bfseries\large}
  {\thesubsection}
  {1em}
  {}

% ======================================================================
%   Header and footer (fancyhdr)
% ======================================================================
\usepackage{fancyhdr}
\pagestyle{fancy} 
\fancyhf{}                          % Clear all header and footer fields

% Left header: current section name in small caps
\fancyhead[L]{\small\textsc{\leftmark}}

% Right header: page number
\fancyhead[R]{\small\thepage}

% Thin horizontal rule under the header
\renewcommand{\headrulewidth}{0.5pt}

% ======================================================================
%   Line spacing
% ======================================================================
\usepackage{setspace}
\onehalfspacing                     % 1.5 line spacing

% ======================================================================
%   Miscellaneous useful packages
% ======================================================================
\usepackage{tocloft}                % Customize table of contents
\usepackage{xcolor}                 % Colors (for code, emphasis, etc.)
\usepackage{booktabs}               % Professional-looking tables

% ======================================================================
%   Source code formatting
% ======================================================================
\usepackage{listings}

% Basic listings configuration
\lstset{
  basicstyle      = \ttfamily\small,     % Monospaced, small size
  backgroundcolor = \color{gray!10},     % Light gray background
  frame           = single,              % Single-line frame around code
  breaklines      = true,                % Wrap long lines
  tabsize         = 2,                   % Optional: tab width
  numbers         = none                 % Optional: omit line numbers
}

% ======================================================================
%   Document body
% ======================================================================
% Full-width abstract (no extra margins)
\usepackage{ragged2e} % for \justifying

% Full-width, justified abstract environment
\newenvironment{wideabstract}{%
  \par\vspace{1em}%
  \noindent\textbf{Abstract}\par
  \vspace{0.2em}%
  \justifying         % <-- full justification, overrides \centering
}{%
  \par\vspace{1em}%
}


% =============================
% Cover-page "function"
% =============================
% Argument #1: abstract text
\newcommand{\MakeCoverPage}[1]{%
  \begin{titlepage}
    \centering
    \vspace*{4cm}
    {\Huge\bfseries \DocTitle\par}
    \vspace{2cm}
    {\Large \DocAuthor\par}
    \vspace{1cm}
    {\large Universidad de las Américas Puebla\par}
    \vfill

    % --- Abstract ---
    \begin{wideabstract}
      #1%
    \end{wideabstract}

    \vspace{0.8cm}
    % Printed keywords: same as pdfkeywords
    {\bfseries Keywords:} \DocKeywords

    \vspace{1cm}
    {\large \today\par}
  \end{titlepage}
}



\begin{document}

% --- Portada ---

\MakeCoverPage{%
  This document provides a template for reports in the "AI in Financial Services"
  course, using EB Garamond for prose and Libertinus Math for formulas. It includes
  a cover page, abstract, table of contents, and sample sections for math and text.
  Additional content demonstrates tables, code, and references.
}




% --- Tabla de contenidos ---
\tableofcontents
\thispagestyle{empty}
\newpage

% --- Inicio del documento ---
\section*{Record of Answers and Brief Explanations (Paradigms, Bets, Dutch Book)}

\subsection*{Question 1}

\textbf{Question.}  
If you randomly guess on a 4-option multiple choice question, you have probability $0.25$ of being correct.  
Which probabilistic paradigm does this argument best demonstrate?

\textbf{Answer.}  
\[
\text{Classical}
\]

\textbf{Explanation.}  
This uses the idea of \emph{equally likely outcomes} (4 options, 1 correct), so
\[
P(\text{correct}) = \frac{1}{4}.
\]
That is the classical (Laplace) definition.

\bigskip

\subsection*{Question 2}

\textbf{Question.}  
On a 3-option question, one option has a keyword the professor used often, so you assign it probability $> 1/3$ of being correct (based on your belief and extra information). Which paradigm?

\textbf{Answer.}  
\[
\text{Bayesian}
\]

\textbf{Explanation.}  
You are using \emph{subjective} beliefs updated by extra information (the keyword) rather than symmetry or long-run frequencies, which is Bayesian in spirit.

\bigskip

\subsection*{Question 3}

\textbf{Question.}  
Empirically, one in three students participates in extracurricular activities, so you conclude
\[
P(\text{random student participates}) = \frac{1}{3}.
\]
Which paradigm?

\textbf{Answer.}  
\[
\text{Frequentist}
\]

\textbf{Explanation.}  
This interprets probability as a \emph{long-run relative frequency} in a population or repeated sampling.

\bigskip

\subsection*{Question 4 — Chess bet, $p=1$}

\textbf{Question.}  
Bet: if \emph{she} wins, you pay her \$3; if \emph{you} win, she pays you \$5.  
If she is 100\% confident she will win, what is her expected return?

\textbf{Answer.}  
\[
3
\]

\textbf{Explanation.}  
Let $p$ be her personal probability of winning. For her,
\[
\text{payoff} =
\begin{cases}
+3, & \text{she wins},\\
-5, & \text{she loses}.
\end{cases}
\]
With $p=1$,
\[
\mathbb{E}[\text{payoff}]
= 1\cdot 3 + 0\cdot(-5) = 3.
\]

\bigskip

\subsection*{Question 5 — Chess bet, $p=0.5$}

\textbf{Question.}  
Same bet, but now her personal probability of winning is $p=0.5$. What is her expected return?

\textbf{Answer.}  
\[
-1
\]

\textbf{Explanation.}  
\[
\mathbb{E}[\text{payoff}]
= p\cdot 3 + (1-p)\cdot (-5)
= 0.5\cdot 3 + 0.5\cdot (-5)
= 1.5 - 2.5 = -1.
\]

\bigskip

\subsection*{Question 6 — Fair bet $\Rightarrow$ her $p$}

\textbf{Question.}  
She will only accept the bet if it is \emph{fair} for her, i.e.\ expected return $=0$.  
Find her personal probability $p$ of winning.

\textbf{Answer.}  
\[
p = \frac{5}{8}.
\]

\textbf{Explanation.}  
As before,
\[
\mathbb{E}[\text{payoff}]
= 3p + (-5)(1-p)
= 3p - 5 + 5p
= 8p - 5.
\]
Set this equal to $0$ for a fair bet:
\[
8p - 5 = 0
\quad\Longrightarrow\quad
p = \frac{5}{8}.
\]

\bigskip

\subsection*{Question 7 — Dutch book payoff}

\textbf{Question.}  
Two bets:

\begin{itemize}
  \item[(i)] If it rains or is overcast tomorrow, you pay him \$4; otherwise he pays you \$6.
  \item[(ii)] If it is sunny tomorrow, you pay him \$5; otherwise he pays you \$5.
\end{itemize}

Events considered: rain, overcast, sunny (mutually exclusive and exhaustive).  
If you take \emph{both} bets, how much do you win regardless of the outcome?

\textbf{Answer.}  
\[
1
\]

\textbf{Explanation.}  
Let us compute your net payoff in each case.

\paragraph{Bet (i).}
\[
\text{rain/overcast: } -4, \quad
\text{sunny: } +6.
\]

\paragraph{Bet (ii).}
\[
\text{sunny: } -5, \quad
\text{not sunny (rain/overcast): } +5.
\]

Total payoff:

\[
\begin{aligned}
\text{Rain/overcast:} &\quad -4 + 5 = +1,\\
\text{Sunny:}         &\quad +6 - 5 = +1.
\end{aligned}
\]

So you win \$1 no matter what happens.

\bigskip

\subsection*{Question 8 — Incoherent probabilities}

\textbf{Question.}  
For bet (i) to be fair, his probability that it rains or is overcast must be $0.6$.  
For bet (ii) to be fair, his probability that it is sunny must be $0.5$.  

These events are exhaustive and disjoint, but his “probabilities” do not sum to $1$.  
What do they sum to?

\textbf{Answer.}  
\[
0.6 + 0.5 = 1.1.
\]

\textbf{Explanation.}  
Let $A =$ “rain or overcast” and $B =$ “sunny.”  
He is using
\[
P(A) = 0.6,\quad P(B) = 0.5,
\]
with $A$ and $B$ disjoint and exhaustive. Coherent probabilities should satisfy
\[
P(A) + P(B) = 1,
\]
but instead
\[
P(A) + P(B) = 0.6 + 0.5 = 1.1 > 1,
\]
which is impossible, and this incoherence enables the Dutch book.

\end{document}
