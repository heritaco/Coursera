\documentclass[10pt,a4paper]{article} % Compila con lualatex o xelatex

% --- Layout & links ---
\usepackage[margin=1.5in]{geometry}
\usepackage[hidelinks]{hyperref}

% --- Fuentes (texto y matemáticas) ---
\usepackage{fontspec}
\setmainfont{EB Garamond}[
    UprightFont = * Medium,
    ItalicFont = * Italic,
    BoldFont = * Bold,
    % ExtraBoldFont = * ExtraBold,
    BoldItalicFont = * Bold Italic,
]
\usepackage{libertinust1math}

% --- Encabezado con nombre de la sección ---
\usepackage{fancyhdr}
\pagestyle{fancy}
\fancyhf{}
\fancyhead[L]{\small\leftmark}
\fancyhead[R]{\small\thepage}
\renewcommand{\headrulewidth}{0.1pt}

% --- Espaciado ---
\usepackage{setspace}
\setstretch{1.2}
\setlength{\parskip}{0.2em}

% --- Otros paquetes útiles ---
\usepackage{titlesec}
\usepackage{tocloft}
\usepackage{booktabs}
\usepackage{listings}
\usepackage{xcolor}

% --- Código fuente ---
\lstset{
  basicstyle=\ttfamily\small,
  backgroundcolor=\color{gray!10},
  frame=single,
  breaklines=true
}

\begin{document}

% --- Portada ---
\begin{titlepage}
  \centering
  \vspace*{4cm}
  {\Huge\bfseries Artificial Intelligence in Algorithmic Trading: An Extended Guide\par}
  \vspace{2cm}
  {\Large Author Name\par}
  \vspace{1cm}
  {\large Universidad de las Américas Puebla\par}
  \vfill
  {\large \today\par}
\end{titlepage}

% --- Abstract ---
\begin{abstract}
This document provides a template for reports in the "AI in Financial Services" course, using EB Garamond for prose and Libertinus Math for formulas. It includes a cover page, abstract, table of contents, and sample sections for math and text. Additional content demonstrates tables, code, and references.
\end{abstract}

% --- Tabla de contenidos ---
\tableofcontents
\thispagestyle{empty}
\newpage

% --- Inicio del documento ---
\section{Overview}
This document is a minimal example using EB Garamond for prose and libertinust1math for formulas.
Links like \href{https://example.com}{this one} are active.

\section{Introduction to Algorithmic Trading}
Algorithmic trading uses computer programs to execute trades at speeds and frequencies that are impossible for humans. Algorithms can be based on timing, price, quantity, or any mathematical model. The use of artificial intelligence (AI) has further enhanced the capabilities of algorithmic trading by enabling adaptive strategies and pattern recognition.

\subsection{Historical Context}
The evolution of algorithmic trading began in the 1970s with the introduction of electronic trading platforms. Over the decades, advancements in computing power and data availability have led to the widespread adoption of automated trading systems.

\subsection{Benefits and Risks}
Algorithmic trading offers several benefits:
\begin{itemize}
  \item Increased speed and accuracy of order execution.
  \item Reduced transaction costs.
  \item Ability to backtest strategies using historical data.
\end{itemize}
However, it also introduces risks such as:
\begin{itemize}
  \item Systematic errors due to bugs in code.
  \item Market instability from high-frequency trading.
  \item Overfitting of models to historical data.
\end{itemize}

\newpage
\section{Math sample}
Let $S_t$ follow a geometric Brownian motion:
\[
  dS_t = \mu S_t\,dt + \sigma S_t\,dW_t,\qquad
  \Rightarrow\;
  S_T = S_0 \exp\!\Big((\mu-\tfrac{1}{2}\sigma^2)T+\sigma\sqrt{T}\,Z\Big).
\]
The Black--Scholes call price:
\[
  C = S_0 e^{-qT}\Phi(d_1)-K e^{-rT}\Phi(d_2),
  \quad d_{1,2}=\frac{\ln(S_0/K)+(r-q\pm\tfrac{1}{2}\sigma^2)T}{\sigma\sqrt{T}}.
\]

This is how the stats operator, like the expected value, variance, and other math functions, look like with the selected font:

\[
  \mathbb{E}[X], \quad \mathbb{V}[X], \quad \mathbb{P}(X \in A)
  % integral and dif
  \int f(x)\,dx, \quad \frac{d}{dx}f(x)
  % other operators
  , \quad \nabla f(x), \quad \partial_x f(x)
\]

\section{Machine Learning in Finance}
Machine learning (ML) techniques are widely used in financial markets for tasks such as price prediction, portfolio optimization, and risk management. Common algorithms include linear regression, decision trees, and neural networks.

\subsection{Example: Linear Regression}
Suppose we want to predict the price of a stock based on historical features. The linear regression model is:
\[
  y = \beta_0 + \beta_1 x_1 + \cdots + \beta_n x_n + \epsilon
\]
where $y$ is the predicted price, $x_i$ are features, $\beta_i$ are coefficients, and $\epsilon$ is the error term.

\newpage
\section{Sample Table}
Table~\ref{tab:strategies} summarizes common algorithmic trading strategies.

\begin{table}[h!]
\centering
\caption{Common Algorithmic Trading Strategies}
\label{tab:strategies}
\begin{tabular}{@{}lll@{}}
\toprule
Strategy & Description & Example \\
\midrule
Trend Following & Buy/sell based on price trends & Moving average crossover \\
Mean Reversion & Trade on price deviations from mean & Pairs trading \\
Market Making & Provide liquidity by quoting prices & Bid-ask spread capture \\
\bottomrule
\end{tabular}
\end{table}

\section{Sample Code}
Below is a Python code snippet for a simple moving average crossover strategy.

\begin{lstlisting}[language=Python, caption={Simple Moving Average Crossover}]
import pandas as pd

def sma_crossover(prices, short=20, long=50):
    prices['SMA_short'] = prices['Close'].rolling(window=short).mean()
    prices['SMA_long'] = prices['Close'].rolling(window=long).mean()
    prices['Signal'] = 0
    prices.loc[prices['SMA_short'] > prices['SMA_long'], 'Signal'] = 1
    prices.loc[prices['SMA_short'] < prices['SMA_long'], 'Signal'] = -1
    return prices
\end{lstlisting}

\section{Conclusion}
Artificial intelligence and machine learning have transformed algorithmic trading, enabling more sophisticated and adaptive strategies. However, practitioners must be aware of the risks and ensure robust testing and validation of their models.

\section{References}
\begin{itemize}
  \item Hull, J. C. (2018). \textit{Options, Futures, and Other Derivatives}. Pearson.
  \item Chan, E. (2013). \textit{Algorithmic Trading: Winning Strategies and Their Rationale}. Wiley.
  \item \href{https://www.investopedia.com/terms/a/algorithmictrading.asp}{Investopedia: Algorithmic Trading}
\end{itemize}

\end{document}
